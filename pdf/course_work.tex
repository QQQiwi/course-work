\documentclass[bachelor, och, coursework]{shiza}
% параметр - тип обучения - одно из значений:
%    spec     - специальность
%    bachelor - бакалавриат (по умолчанию)
%    master   - магистратура
% параметр - форма обучения - одно из значений:
%    och   - очное (по умолчанию)
%    zaoch - заочное
% параметр - тип работы - одно из значений:
%    referat    - реферат
%    coursework - курсовая работа (по умолчанию)
%    diploma    - дипломная работа
%    pract      - отчет по практике
% параметр - включение шрифта
%    times    - включение шрифта Times New Roman (если установлен)
%               по умолчанию выключен
\usepackage{subfigure}
\usepackage{tikz,pgfplots}
\pgfplotsset{compat=1.5}
\usepackage{float}

%\usepackage{titlesec}
\setcounter{secnumdepth}{4}
%\titleformat{\paragraph}
%{\normalfont\normalsize}{\theparagraph}{1em}{}
%\titlespacing*{\paragraph}
%{35.5pt}{3.25ex plus 1ex minus .2ex}{1.5ex plus .2ex}

\titleformat{\paragraph}[block]
{\hspace{1.25cm}\normalfont}
{\theparagraph}{1ex}{}
\titlespacing{\paragraph}
{0cm}{2ex plus 1ex minus .2ex}{.4ex plus.2ex}

% --------------------------------------------------------------------------%


\usepackage[T2A]{fontenc}
\usepackage[utf8]{inputenc}
\usepackage{graphicx}
\graphicspath{ {./images/} }
\usepackage{tempora}

\usepackage[sort,compress]{cite}
\usepackage{amsmath}
\usepackage{amssymb}
\usepackage{amsthm}
\usepackage{fancyvrb}
\usepackage{listings}
\usepackage{listingsutf8}
\usepackage{longtable}
\usepackage{array}
\usepackage[english,russian]{babel}

% \usepackage[colorlinks=true]{hyperref}
\usepackage{url}

\usepackage{underscore}
\usepackage{setspace}
\usepackage{indentfirst} 
\usepackage{mathtools}
\usepackage{amsfonts}
\usepackage{enumitem}
\usepackage{tikz}

\newcommand{\eqdef}{\stackrel {\rm def}{=}}
\newcommand{\specialcell}[2][c]{%
\begin{tabular}[#1]{@{}c@{}}#2\end{tabular}}

\renewcommand\theFancyVerbLine{\small\arabic{FancyVerbLine}}

\newtheorem{lem}{Лемма}

\begin{document}

% Кафедра (в родительном падеже)
\chair{теоретических основ компьютерной безопасности и криптографии}

% Тема работы
\title{Обнаружение объектов на изображении искусственными нейронными сетями}

% Курс
\course{2}

% Группа
\group{231}

% Факультет (в родительном падеже) (по умолчанию "факультета КНиИТ")
\department{факультета КНиИТ}

% Специальность/направление код - наименование
%\napravlenie{09.03.04 "--- Программная инженерия}
%\napravlenie{010500 "--- Математическое обеспечение и администрирование информационных систем}
%\napravlenie{230100 "--- Информатика и вычислительная техника}
%\napravlenie{231000 "--- Программная инженерия}
\napravlenie{100501 "--- Компьютерная безопасность}

% Для студентки. Для работы студента следующая команда не нужна.
% \studenttitle{Студентки}

% Фамилия, имя, отчество в родительном падеже
\author{Улитина Ивана Владимировича}

% Заведующий кафедрой
% \chtitle{} % степень, звание
% \chname{}

%Научный руководитель (для реферата преподаватель проверяющий работу)
\satitle{доцент} %должность, степень, звание
\saname{Слеповичев Иван Иванович}

% Руководитель практики от организации (только для практики,
% для остальных типов работ не используется)
% \patitle{к.ф.-м.н.}
% \paname{С.~В.~Миронов}

% Семестр (только для практики, для остальных
% типов работ не используется)
%\term{8}

% Наименование практики (только для практики, для остальных
% типов работ не используется)
%\practtype{преддипломная}

% Продолжительность практики (количество недель) (только для практики,
% для остальных типов работ не используется)
%\duration{4}

% Даты начала и окончания практики (только для практики, для остальных
% типов работ не используется)
%\practStart{30.04.2019}
%\practFinish{27.05.2019}

% Год выполнения отчета
\date{2021}

\maketitle

% Включение нумерации рисунков, формул и таблиц по разделам
% (по умолчанию - нумерация сквозная)
% (допускается оба вида нумерации)
% \secNumbering

%-------------------------------------------------------------------------------------------

\tableofcontents

\intro

    В современном мире, в эпоху информационных технологий, с каждым годом увеличивается количество задач, которые могут быть решены с помощью компьютера. И чем сложнее компьютер становится, тем более серьезные проблемы становятся решаемыми. Однако были такие задачи, которые легко решались только человеческим разумом, за счет интуиции, а не вычислительной машиной, так как лучше всего компьютером разрешались такие задачи, которые имели математическую составляющую, подход к которым осуществлялся с помощью определенного набора аксиом и правил. Совсем недавно, относительно появления компьютера, начала развиваться наука об искусственном интеллекте, которая открыла компьютерам возможность находить решения для таких проблем, которые раньше могли быть решены только человеком. Вследствии всего этого сейчас активно начала использоваться технология нейронных сетей, которые позволяют решать простые человеческие проблемы достаточно быстро, упрощая повседневную жизнь людей. В качестве примера таких задач можно привести: определение самого короткого маршрута из точки А в точку Б с учетом трафика в городе, перевод с одного языка на другой, автоматическое определение диагноза пациента при наборе симптомов, парковка машины с помощью искусственного интеллекта, сведение звука в музыкальных произведениях, классификация большого объема данных каких-либо форматов по различным ключам, определение объектов на изображении и многое другое.\\
    В данной работе будет рассматриваться задача обнаружения объектов на изображении посредством искусственных нейронных сетей, её примеры в реальной жизни, математическая составляющая её решения и анализ соответствующих этому решению инструментов.

\section{Концепция технологии}

    Каждый отдельный элемент информации, включаемый в представление о каком-либо анализируемом объекте, называется \textbf{признаком}. Большая часть задач искусственного интеллекта решается в два этапа: корректный подбор признаков, а затем их передача алгоритму машинного обучения. В качестве примера можно взять задачу идентификации объекта по звуку речи. В ней полезным признаком является речевой тракт или же голосовой диапазон. Он
    позволяет с большой точностью определить, является ли говорящий объект мужчиной, женщиной или
    ребенком.
    Но далеко не во задачах можно сразу понять, какие признаки стоит выделять: для этого достаточно рассмотреть ситуацию, в которой необходимо написать программу обнаружения автомобилей на фотографиях. Известно, что у автомобилей есть колеса, вследствии чего можно в качестве одного из признаков выбрать наличие колеса. Однако нельзя сказать, что описание колеса на уровне пикселей легкая задача.
    Колесо характеризуется простой геометрической формой, но его распознавание на изображении
    нередко бывает осложнено различными факторами, такими как отбрасывание теней, наличие щитка для защиты колеса от грязи, объекты на переднем плане, которые могут закрывать часть колеса, и т. д.

    В качестве решений подобной задачи рассматривают использование машинного обучения не только в качестве способа нахождения отображения представления на результат, но и для определения самого представления. Такой подход к реализации машинного обучения называется \textbf{обучением представлений}. С помощью этих представлений, которые являются результатом обучения, получается более точно определить объект на изображении, чем с помощью представлений, созданных вручную. В качестве характерного преимущества подобных реализаций стоит отметить быструю адаптацию ИИ к новым задачам с учетом минимального вмешательства человека в изменение структуры конкретной нейросети. 

    Важную роль в алгоритме обучения представлений является понятие \textbf{автокодировщика} - совокупность функции кодирования (которая преобразует входные данные в удобное для решения задачи представление) и функции декодирования, являющейся обратной по смыслу к предыдущей функции. Обучение автокодировщиков устроено таким образом, что при каждой последующей итерации, в ходе которой происходит кодирование и декодирование некоторой информации, каждое новое представление обладало всё большим количеством полезных свойств при наименьшей потери обрабатываемой информации.
\section{Применение технологии}

    
\section{Инструменты для реализации}

\conclusion

\end{document}
